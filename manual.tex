\documentclass{book}

\usepackage[activate={true,nocompatibility}
           ,final
           ,tracking=true
           ,kerning=true
           ,spacing=true
           ,factor=1100
           ,stretch=10
           ,shrink=10]{microtype}
\microtypecontext{spacing=nonfrench}

\usepackage{verbatim}
\newenvironment{code}{\footnotesize\verbatim}{\endverbatim\normalsize}

\title{Tupa Specification \& Ibew Reference Implementation, with Manual}
\author{blank}
\date{\today}

\begin{document}
\maketitle
\tableofcontents

\part{Introduction and User's Manual}
Ibew is the reference implementation of the Tupa specification.
Tupa is a specification for an instant messaging program, which lists in detail features that serve the unique needs of plural people.
Ibew is an application which implements those features, and should be used as a reference for any ambiguity in the specification.
This document contains both the specification, and the application; this naturally divides the document into two parts: the test suite, which exhaustively explains and formalizes Tupa, and the implementation, which will .


\part{Tupa: Specification}

Tupa differs from standard instant messenger applications with one major feature: accounts are not directly tied to the titles that appear on messages.
Outside of that, it is a fairly standard instant messenger application.
There are two layers of indirection for organizing messages; the top layer is called systems, which contain channels, which contain messages.
Messages are rich text associated with a specific headmate, with the ability to add attachments.
By associating messages with a headmate, we finally get into the unique feature.
Instead of an account having a single name and appearance which distinguish their messages, accounts can have many names and appearances, which are called `headmates.'

This specification will be split up into a series of explanations of different features, interspersed with programs that will test an implementation's correctness.
This ensures that anyone developing a Tupa implementation will be able to ensure they interoperate correctly with other implementations.
To begin with, we'll go over the different categories we sort features into, broadly stated as user stories.

\begin{enumerate}
\item A prospective user wishes to sign up.
\item An existing user wishes to sign in.
\item A user reads existing messages.
\item A user sends a new message.
\item A user changes their settings.
\item A user modifies their headmates.
\item An administrator promotes a moderator.
\item A moderator uses their unique priveleges.
\end{enumerate}

\input{test/Spec.lhs}

\part{Ibew: Implementation}
\input{src/App.lhs}
\input{src/DB.lhs}
\input{src/Endpoints/Landing.lhs}
\input{src/Hashing.lhs}
\input{src/Endpoints/Login.lhs}
\input{src/Endpoints/Register.lhs}
\input{src/Endpoints/Messages.lhs}
\input{src/Configuration.lhs}
\input{app/Main.lhs}
\end{document}
